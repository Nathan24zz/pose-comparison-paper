% Ubah judul dan label berikut sesuai dengan yang diinginkan.
\section{Related Works}
\label{sec:relatedworks}

% Ubah paragraf-paragraf pada bagian ini sesuai dengan yang diinginkan.

Previous studies have succeeded in creating a model that can obtain multi-robot pose estimation from images.
This research was conducted by \citet{amini2021} from the University of Bonn using the bottom-up approach method.
The bottom-up approach is a method that detects body joints and groups them into individuals simultaneously.
However, the dataset that used only for adult-size category robots and a few for robots without skin. Therefore,
we will add data from the Ichiro robot, where the Ichiro robot is a robot category teen size and type of robot without skin.

In addition, as done by \citet{güneysu2017} who developed an assistive robot as a physical trainer for young children.
They have designed and implemented a fully autonomous human-robot interaction system with social assistance that can increase a
child's involvement in a variety of physical exercise by providing real-time feedback that they get from the IMU (Inertial Measurement Units) worn on the child.
However, in contrast to previous research, this research does not use any sensors that are worn by the user. This will most likely increase users' convenience during physical exercise.