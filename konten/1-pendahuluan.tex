% Ubah judul dan label berikut sesuai dengan yang diinginkan.
\section{Introduction}
\label{sec:introduction}

% Ubah paragraf-paragraf pada bagian ini sesuai dengan yang diinginkan.

Robots have experienced significant development over the last few years 
because of their ability to perform multiple tasks quickly and precisely.
One form of development is socially assistive robots (SARs). 
SARs are a type of robot that combines the aspects of assistive robotics (AR)
and socially interactive robotics (SIR), so it makes SARs a robot capable of providing assistance to users in the form of social interaction \citep{feil2005}.

Regular physical activity is a central protective factor for health.
A report from WHO says that lack of physical activity contributes to around 3.2 million premature deaths each year worldwide.
The research also shows that regular exercise can help older adults improve physical fitness, immune system, sleep quality, stress levels, and overcome other health problems \citep{lotfi2018}.
However, motivation to engage in physical activity declines with age. 
This is due to understaffed and high costs of personal trainers as well as the elderly who cannot be permanently motivated and instructed to engage in physical activity.
Based on research conducted by \citet{ruf2020} it can be concluded that the use of humanoid robot can motivate older people to carry out regular physical activity.

The application of SARs is very diverse, for example, humanoid robot that become physical trainers for children \citep{güneysu2017}, 
robotic systems for physical training of the elderly \citep{avioz2021}, and so on. The methods used also vary, for instance, 
you can put sensors on the user and get feedback, then provide a response based on the data obtained through the sensor \citep{güneysu2017}. 
Another method is to use pose estimation obtained through \emph{deep learning} models.
However, previous studies have only compared the angle and depth of each keypoint from estimated human pose within certain boundaries or compared the results of human poses with the poses that are used as a guide \citep{romeo}.
There is still little study that directly compares the suitability between poses performed by human and robot.

For that, on this occasion, we propose study related to mimicking between humanoid robot and human based on real-time pose estimation. 
In addition to finding the best pose estimation method for both humanoid robot and human, this study will also compare poses between them and make a web for controlling their interaction.

Pembahasan pada paper ini dimulai dengan presentasi mengenai penelitian lain (Bagian \ref{sec:relatedworks}).
Kemudian dilanjutkan dengan penjelasan mengenai arsitektur dari sistem yang dibuat (Bagian \ref{sec:poseestimation}).
Berdasarkan hal tersebut, kami menunjukkan lorem ipsum (Bagian \ref{sec:loremipsum}).
Terakhir, didapatkan kesimpulan dari penelitian yang telah dilakukan (Bagian \ref{sec:kesimpulan}).
