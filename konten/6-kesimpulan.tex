% Ubah judul dan label berikut sesuai dengan yang diinginkan.
\section{Conclusion}
\label{sec:conclusion}

% Ubah paragraf-paragraf pada bagian ini sesuai dengan yang diinginkan.

\begin{enumerate}

    \item The creation of a new dataset has been completed by merging the HumanoidRobotPose dataset and Ichiro's poses dataset 
          so the size of the robots become varies (adult, teen, and kid size).
    \item Annotation tool that is used to make Ichiro's pose dataset is coco-annotator because this is the only tool that the writer found that can export the datasets to COCO Format correctly.
    \item The Mediapipe was the chosen method for human pose estimation based on our needs of this study and its time inference.
    \item RCNN Keypoint is capable and reliable for detecting humanoid robot with 6 keypoints  based on results on the test set and detection results
          although it can not be performed in real-time even after converting to OpenVINO.
    \item Cosine Similarity is a suitable method to compare the suitability between humanoid robot pose with human poses because the result is higher when
          humans make a movement that is more like a robot and vice versa.
  
\end{enumerate}