% Ubah judul dan label berikut sesuai dengan yang diinginkan.
\section{Conclusion}
\label{sec:conclusion}

% Ubah paragraf-paragraf pada bagian ini sesuai dengan yang diinginkan.

\begin{enumerate}

      \item The creation of a new dataset has been completed by merging the HumanoidRobotPose dataset and Ichiro's poses dataset 
            so the size of the robots become varies (adult, teen, and kid size). Most of the additions are one robot instance per image with large scale (according to COCO dataset).
      \item RCNN Keypoint is capable and reliable for detecting humanoid robot with 6 keypoints because it outperforms other models with \emph{AP} 0.879 and \emph{AR} 0.925 on the test set (except for AP medium).
      \item Cosine Similarity is a suitable method to compare the suitability between humanoid robot pose with human poses because the result is higher when
            human make a movement that is more like a robot's motion and vice versa.
\end{enumerate}