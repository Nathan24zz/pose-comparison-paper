% Ubah judul dan label berikut sesuai dengan yang diinginkan.
\section{Conclusion}
\label{sec:conclusion}

% Ubah paragraf-paragraf pada bagian ini sesuai dengan yang diinginkan.

\begin{enumerate}

      \item The creation of a new dataset has been completed by merging the HumanoidRobotPose dataset and Ichiro's poses dataset 
            so the size of the robots become varies (adult, teen, and kid size). Furthermore, most of the additions (Ichiro's pose dataset) are one robot instance and
            few are two instances per image with large scale (according to COCO dataset).
      \item RCNN Keypoint is capable and reliable for detecting humanoid robot with 6 keypoints because it outperforms other models in \emph{AP} and \emph{AR} results on the test set (except for AP medium)
            and shows the best detection results although it can not be performed in real-time even after converting to OpenVINO.
      \item Cosine Similarity is a suitable method to compare the suitability between humanoid robot pose with human poses because the result is higher when
            human make a movement that is more like a robot's motion and the result is lower when human make a movement that is less like robot's motion.
\end{enumerate}